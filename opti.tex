%% Le document à compléter en LaTeX
%


%

% !TEX TS-program = pdflatex
% !TEX encoding = UTF-8 Unicode

% % !TEX %root


\documentclass[10pt, a4, oneside, headings=normal]{scrreprt}

\usepackage[utf8]{inputenc}
\usepackage[T1]{fontenc}
\usepackage{lmodern}
\usepackage[np]{numprint} % affichage correct des nombres : \numprint{} > \np{}
\usepackage{xspace}
\usepackage{graphicx}
\usepackage{color}
\usepackage{xcolor}
\usepackage{microtype}
\usepackage{textcomp}
\usepackage[hidelinks]{hyperref}
\usepackage{longtable}

\usepackage{booktabs} % for much better looking tables
\usepackage{array} % for better arrays (eg matrices) in maths
\usepackage{paralist} % very flexible & customisable lists (eg. enumerate/itemize, etc.)
\usepackage{verbatim} % adds environment for commenting out blocks of text & for better verbatim
\usepackage{subfig} % make it possible to include more than one captioned figure/table in a single float

\usepackage{amsmath}
\usepackage{listings}

\definecolor{mygreen}{rgb}{0,0.6,0}
\definecolor{mygray}{rgb}{0.5,0.5,0.5}
\definecolor{mymauve}{rgb}{0.58,0,0.82}

\lstset{ %
backgroundcolor=\color{white}, % choose the background color; you must add \usepackage{color} or \usepackage{xcolor}
basicstyle=\footnotesize, % the size of the fonts that are used for the code
breakatwhitespace=false, % sets if automatic breaks should only happen at whitespace
breaklines=true, % sets automatic line breaking
captionpos=b, % sets the caption-position to bottom
commentstyle=\color{mygreen}, % comment style
%deletekeywords={...}, % if you want to delete keywords from the given language
%escapeinside={\%*}{*)}, % if you want to add LaTeX within your code
%extendedchars=true, % lets you use non-ASCII characters; for 8-bits encodings only, does not work with UTF-8
frame=single, % adds a frame around the code
keepspaces=true, % keeps spaces in text, useful for keeping indentation of code (possibly needs columns=flexible)
keywordstyle=\color{blue}, % keyword style
language=Java, % the language of the code
%morekeywords={*,...}, % if you want to add more keywords to the set
numbers=left, % where to put the line-numbers; possible values are (none, left, right)
numbersep=5pt, % how far the line-numbers are from the code
numberstyle=\tiny\color{mygray}, % the style that is used for the line-numbers
rulecolor=\color{black}, % if not set, the frame-color may be changed on line-breaks within not-black text (e.g. comments (green here))
showspaces=false, % show spaces everywhere adding particular underscores; it overrides 'showstringspaces'
showstringspaces=true, % underline spaces within strings only
showtabs=false, % show tabs within strings adding particular underscores
stepnumber=5, % the step between two line-numbers. If it's 1, each line will be numbered
firstnumber=0,
stringstyle=\color{mymauve}, % string literal style
tabsize=4, % sets default tabsize to 2 spaces
% title=\lstname % show the filename of files included with \lstinputlisting; also try caption instead of title
}

\newcommand{\javaprogram}[1]{\subsection{\texttt{#1}}\lstinputlisting{RATTA/#1}}

\renewcommand{\arraystretch}{1,2}

\usepackage[frenchb]{babel}

\begin{document}
%\maketitle
\begin{titlepage}

\fontfamily{phv}\selectfont
\begin{center}
LINMA 1702
\end{center}
\vspace*{\stretch{0.1}}
\begin{center}\huge\bfseries
Projet
\end{center}

\begin{center}
Planification de la production d'une ligne d'assemblage de smartphones
\end{center}
\hrule

\begin{center}\large
\end{center}
\begin{center}\large
\end{center}
\begin{center}\large
\end{center}
\begin{center}\large
\end{center}
\begin{center}\large
\end{center}
\begin{center}\large
\end{center}
\begin{center}\large
\end{center}



\begin{center}\large
\end{center}

\begin{center}\large
 François \bsc{Brancaleone} (58671300) \\  Mathieu \bsc{Delandmeter} (62401300) \\Harold \bsc{Della Faille} (??)
\end{center}

\begin{center}\large
\end{center}

\vspace*{\stretch{2}}
\begin{center}
Ann\'ee acad\'emique 2014 - 2015
\end{center}
\end{titlepage}


%%%%%% LE DOCUMENT COMMENCE ICI %%%%%%%%

\tableofcontents

\section{Modélisation de la ligne d'assemblage}
\subsection{Question 1}
\emph{Donnez une formulation linéaire (continue, sans variables entières) du problème de la planification de la ligne d'assemblage à personnel constant. Décrivez successivement variables,
contraintes et fonction objectif.}
\\

Comme toute entreprise qui désire réaliser des bénéfices, nous souhaitons minimiser le coût que représentera la production des smartphones.
Voici la liste des coûts nécessaires à la production d'un seul smartphone :
\begin{enumerate}
\item cout-materiaux
\item cout
\end{enumerate}

\subsection{Question 2}
\emph{Démontrez que, sous certaines hypothèses raisonnables, il est possible de garantir que votre modèle linéaire admette toujours une solution entière, c'est-à-dire ne comportant que des quantités à produire entières. L'une de ces hypothèses est l'intégralité de la quantité à produire chaque semaine; quelles sont les autres ? }

\end{document}
